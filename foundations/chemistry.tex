\chapter{Chemistry} \label{chem_foundations}

\section{Atomic Properties and Periodic Trends}

Exploring some of the periodic trends are useful to understand the emergent properties of molecular interactions. Some of the relevant properties for biochemistry are electronegativity, atomic radius, polarizability, and ionization energy. 

Elemental fluorine is one of the most reactive elements in part due to its high electronegativity (the highest of all elements). This property also results in its small atomic radius (the smallest except for hydrogen, smaller even than Helium which has 5 less electrons). This is because as we move left to right on the periodic table, each added electron in the 2s-2p orbitals have the same energy, while each additional proton adds to the “effective nuclear charge”, pulling the electrons in closer to the nucleus. This also results in its small polarizability.

It is worth noting that the ions of different elements will have a different radius than their elemental form. Cations have smaller atomic radii, since they lose electrons, whereas anions have larger atomic radii, since they add electrons.


\includesvg[scale=0.4]{images/Atomic_&_ionic_radii.svg}

\section{Molecular Structure and Modeling Theories}

The simplest theory you may have heard of is the Lewis electron-pair theory, which we almost never use so I will not review it.

\subsection{The VSEPR Model}
\href{https://chem.libretexts.org/Bookshelves/General_Chemistry/Map%3A_Chemistry_-_The_Central_Science_(Brown_et_al.)/09%3A_Molecular_Geometry_and_Bonding_Theories/9.02%3A_The_VSEPR_Model}{9.2: The VSEPR Model - Chemistry LibreTexts}
The valence-shell electron-pair repulsion (VSEPR) theory is a theory slightly more advanced than the Lewis electron-pair theory you likely learned in high school that can predict the geometry of a wide variety of molecules. This theory does not predict the energies associated with different kinds of bonds and geometries, only the geometry itself. Although simpler than valence bond theory, VSEPR is able to correctly predict many of the geometries.

\subsection{Valence Orbital Theory}
\href{https://chem.libretexts.org/Bookshelves/General_Chemistry/Map%3A_Chemistry_-_The_Central_Science_(Brown_et_al.)/09%3A_Molecular_Geometry_and_Bonding_Theories/9.04%3A_Covalent_Bonding_and_Orbital_Overlap}{9.4: Covalent Bonding and Orbital Overlap - Chemistry LibreTexts}
\href{https://chem.libretexts.org/Bookshelves/General_Chemistry/Map%3A_Chemistry_-_The_Central_Science_(Brown_et_al.)/09%3A_Molecular_Geometry_and_Bonding_Theories/9.05%3A_Hybrid_Orbitals}{9.5: Hybrid Orbitals - Chemistry LibreTexts}

However, the theory that comes up more often in molecular dynamics is valence bond theory which accounts for the formation of hybrid atomic orbitals. This theory uses a knowledge of quantum and electron orbitals to improve upon the VSEPR theory by predicting bond order and relative bonding energetics in addition to geometries, and can also explain why some elements (such as phosphate and sulfur) are able to form more bonds than predicted by the simpler theories.

This is the theory responsible for the labels of “sp2, sp3”, etc on the atom types you might see in a force field description. These refer to atomic orbitals that are formed between the s-orbital of one atom and the p-orbital of the other. If a carbon atom is labeled sp3, this means that all three of its p-orbitals have been combined with the s orbital to form 4 energetically equivalent sp3 orbitals. This results in the tetrahedral geometry we are familiar with. If a carbon atom is labeled sp2, only two of the p-orbitals have been hybridized, leaving one p-orbital available to form a pi bond. 
\href{1.10: Hybridization of Nitrogen, Oxygen, Phosphorus and Sulfur - Chemistry LibreTexts}{1.10: Hybridization of Nitrogen, Oxygen, Phosphorus and Sulfur - Chemistry LibreTexts}

As a mental check, consider the hybridization of nitrogen in pyrrole, pyridine, and piperidine. Which orbital is the lone pair of electrons? How is this related to why the pKa of pyrrole is 3.8, while the pKa of pyridine is 5.23? 
An interesting deep dive here...

Molecular Orbital Theory
1.11: The Nature of Chemical Bonds- Molecular Orbital Theory - Chemistry LibreTexts
9.6: Multiple Bonds - Chemistry LibreTexts
9.7: Molecular Orbitals - Chemistry LibreTexts
There


Density Functional Theory
Density functional theory across chemistry, physics and biology - PMC.
